\chapter{序章}

眨眼间我在清华园已度过了三年。
三年我吃过了园子里11个食堂,出入过4个图书馆,踏遍了6个教学楼,还借着阿甘跑步的时候跑遍了各种各样的小角落,自以为已对清华园非常熟悉了解。
然而,第四年,我却惊讶地发现,园子里还有这么多地方,我未曾踏足……

\vfill

\paragraph{记事}
2020年由于突发的疫情,在家里困了一个学期,6月份终于允许毕业生返校。
笔者毕业后就要去其他学校读博了,因此这本来是能够留在园子里的最后一个学期,只是如今仅剩下寥寥的数周了。
此番离去,之后回来的机会便少了许多。为了能够多留下些回忆,笔者约了朋友C君再在最后离校前逛一逛清华园。
晚上走在人迹罕至的小路上,就会有一种微妙的紧张与刺激感,于是便和C君聊起了魑魅魍魉、月黑风高的事情,谈笑间产生了杜撰些怪谈的想法。
笔者文笔枯燥乏力,实在缺乏表现力,本来只当个玩笑,但终于还是决定写下来,权做四年清华生活的纪念。
此间故事,皆为杜撰;怪力乱神,洵有其缘;茶余闲谈,供君一笑耳。

% 简要提纲
% 
% 1. 素材
%    * 荷塘南面,西湖西北,小山上,月黑杀人夜,风高放火天,零零阁
%    * 荷塘仿若小提琴声的荷香,荷仙的歌声,荷塘月色
%    * 装满水的西湖,服务看不见的客人
%    * 荷塘东侧的小山
%    * 夜晚的工字厅,门口的灯,冥府之门
%    * 苏世民书院后的凉风
%    * 二教的地下室,不开放自习,旧人体实验室
%    * 大礼堂后的小径,限时开放,大礼堂的地下室
